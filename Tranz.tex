\input{text/diss}

\begin{document}

\def\labauthors{Карусевич А.А., Шиков А.П.}
\def\labgroup{440}
\def\labnumber{2}
\def\labtheme{Измерение статистических характеристик полевого транзистора}
\renewcommand{\vec}{\mathbf}
\renewcommand{\phi}{\varphi}
\renewcommand{\hat}{\widehat}

\input{text/titlepage}

\section{Введение}
Полевыми транзисторами (ПТ) называются полупроводниковые приборы, работа которых основана на управлении размерами токопроводящей области (канала) посредством изменения напряженности поперечного электрического поля. Структура:
\begin{figure}[h!]
	\centering
	\includegraphics[width=0.5\linewidth]{fig/fig1.jpg}
	\caption{Структурная схема полевого транзистора}
	\label{fig:1}
\end{figure}

Ток прибора протекает по полупроводниковой области, называемой каналом. В ПТ используется движение носителей заряда только одного знака, поэтому прибор иногда называют униполярным транзистором. Они "истекают" из электрода, называемого истоком, движутся вдоль канала и "стекают" в сток. Затвор транзистора является управляющим электродом. Электрическое поле, возникающее при подаче напряжения между истоком и затвором,Ю изменяет проводимость канала и, следовательно, ток через канал. Таким образом. ток прибора изменяется за счет электрического поля, направленного перпендикулярно движению носителей в канале (поэтому транзистор называют полевым). Носители движутся от истока к стоку под действием продольного электрического поля, создаваемого напряжением между истоком и стоком. Подложка -конструктивный слой полупроводника, который вырезают из полупроводникового слитка, шлифуют и полируют, а затем используют как основу для выращивания на нем тонких эпитаксиальных полупроводниковых слоев, из которых формируется структура приборов.

Существует два основных типа ПТ, различающихся физической структурой и способом управления проводимостью канала. В ПТ с изолированны затвором между металлическим затвором и каналом расположен слой диэлектрика так, что образуется структура металл/диэлектрик/полупроводник. Поэтому такие транзисторы называются МДП-транзисторами. Поперечное электрическое поле, проникая через слой диэлектрика, управляет концентрацией носителей заряда в полупроводниковом канале. Такие транзисторы делятся, в свою очередь на приборы со встроенным и индуцированным каналом. В ПТ с управляющим переходом металлический электрод затвора образует с приповерхностным слоем полупроводника выпрямляющий контакт, на который в рабочем режиме подается обратное напряжение (изоляция затвора путем запирания выпрямляющего контакта). В качестве управляющего перехода могут использоваться p-n переход, гетеропереход или контакт Шоттки.

Все ПТ различают также по типу проводимости канала: транзисторы с каналом p и n-типа. Полярности рабочих напряжений смещения, подаваемых на электроды этих транзисторов, противоположны.

Характерным для всех ПТ является очень малый ток в цепи затвора, так как затвор либо изолирован диэлектриком, либо образует с каналом управляющий переход, включаемый в обратном направлении. Так как затвор в электрических схемах является входным электродом, то ПТ обладает высоким входным сопротивлением на постоянном токе. В этом заключается отличие ПТ от биполярных транзисторов, во входной цепи которых (обычно базовой) протекает значительный ток при прямом напряжении на переходе эмиттер-база. Поэтому входное сопротивление биполярных транзисторов относительно мало. 

В связи с указанным различием входных сопротивлений говорят, что ПТ - это прибор, управляемый напряжением (электрическим полем), а биполярный - управляемый током. В приборах, управляемых напряжением, напряжение на входном электроде из-за высокого входного сопротивления $R_\text{ВХ}$ практически не зависит от параметров самого прибора и определяется ЭДС генератора входного сигнала, если $R_\text{ВХ} \gg R_\text{ГЕН}$, где $R_\text{ГЕН}$ - внутреннее сопротивление генератора. В приборах, управляемых током, входной ток из-за малого входного сопротивления слабо зависит от параметров прибора и определяется током генератора входного сигнала (при $R_\text{ВХ} \ll R_\text{ГЕН}$).

\section{Полевые транзисторы с управляющим переходом}
\subsection{Структура и принцип действия}
В полевых транзисторах с управляющим переходом изменение потока основных носителей происходит с помощью выпрямляющего электрического перехода, смещенного в обратном направлении.

Полевой транзистор с управляющим p-n переходом имеет два омических контакта к области, по которой проходит управляемый ток основных носителей заряда, один управляющий электронно-дырочный переход и один изолирующий от подложки p-n переход. На оба перехода подаются запирающие напряжения
\begin{figure}[h!]
	\centering
	\includegraphics[width=0.5\linewidth]{fig/fig2.jpg}
	\caption{Полевой транзистор с управляющим p-n переходом: а) реальная структура транзистора; б) схема включения модельной (симметричной) структуры транзистора по схеме с общим истоком; в) графическое обозначение полевого транзистора с управляющим переходом. Обозначения: И-исток; З-затвор; С-сток; n-область канала; $p^+$ - область затвора p-типа (значок "+" обозначает высокий уровень содержания легирующей примеси); пунктиром показана обедненная электронами область p-n переходов затвора в слое канала}
	\label{fig:2}
\end{figure}

Управление током стока происходит при изменении обратного напряжения на p-n переходе затвора. При этом изменяется толщина p-n перехода и, следовательно, толщина канала - области, по которой протекает управляемый ток. В связи с тем, что обратные токи малы, мощность, затрачиваемая для управлении током стока, оказывается также малой. Поэтому полевой транзистор обеспечивает усиление сигнала по мощности, току и напряжению. Принцип действия полевого транзистора аналогичен вакуумному триоду. Исток в полевом транзисторе подобен катоду, затвор - сетке, сток - аноду.

\subsection{Статические характеристики}
Важнейшими семействами статистических характеристик ПТ являются выходные статистические характеристики и характеристики передачи.

Выходные статистические характеристики представляют собой зависимость тока стока $J_C$ от напряжения на стоке относительно истока $U_\text{СИ}$ при различных постоянных напряжениях на затворе $U_\text{ЗИ}: J_C=f(U_\text{СИ})$ при $U_\text{ЗИ} = const$:
\begin{figure}[h!]
	\centering
	\includegraphics[width=0.5\linewidth]{fig/fig3.jpg}
	\caption{Выходные статистические ВАХ ПТ при различных напряжениях исток-затвор}
	\label{fig:3}
\end{figure}

На начальном (крутом) участке отличие от линейной зависимости объясняется увеличением толщины управляющего p-n перехода около стока, при котором поперечное сечение канала уменьшается, а сопротивление канала увеличивается. Другая причина нелинейности выходной характеристики - уменьшение подвижности носителей заряда в канале при увеличении в нем напряженности электрического поля.

При $U_\text{СИ} = U_\text{СИ.нас.}$ происходит смыкание ОПЗ затворов сначала около стока. при дальнейшем увеличении $U_\text{СИ}$ длина перекрытой части канала увеличивается, причем статическое сопротивление канала растет пропорционально увеличению напряжения стока, поэтому ток стока остается постоянным (насыщенным).

При увеличении $|U_\text{ЗИ}|$ уменьшается исходное поперечное сечение канала. Поэтому начальные участки выходных характеристик имеют меньший наклон, что соответствует большему начальному статическому сопротивлению канала. Перекрытие канала происходит при меньших напряжениях насыщения.

При некотором напряжении на стоке возникает пробой p-n перехода затвора, обратное напряжение на котором изменяется вдоль канала, достигая максимума у стокового конца. Напряжение, приложенное к p-n переходу затвора в этом месте равно $U_\text{СИ}+|U_\text{ЗИ}|$. Поэтому чем больше $|U_\text{ЗИ}|$, тем меньше $U_\text{СИ.проб.}$

Характеристики передачи представляют собой зависимость $J_C=f(U_\text{ЗИ})$ при $U_\text{СИ} = const$:
\begin{figure}[h!]
	\centering
	\includegraphics[width=0.5\linewidth]{fig/fig4.jpg}
	\caption{Статическая передаточная ВАХ}
	\label{fig:4}
\end{figure}

Они снимаются в режиме насыщения тока стока (пологая часть выходных статических характеристик), т.к. это основной рабочий режим полевого транзистора. При изменении $U_\text{СИ}$ смещением характеристик передачи практически можно пренебречь. При $U_\text{ЗИ} = U_\text{ЗИ.отс} J_C=0$

\section{Экспериментальная часть}
% \subsection{Экспериментальная установка}
% В состав установки для измерения статических характеристик полевого транзистора (рис. 6) производится на установке, состоящей из блока режимов (1) и высокочастотного генератора (2).
% % \begin{figure}[h!]
% % 	\centering
% % 	\includegraphics[width=0.6\linewidth]{fig/fig6.jpg}
% % 	\caption{}
% 	\label{fig:6}
% \end{figure}
Схема:
% \begin{figure}[h!]
% 	\centering
% 	\includegraphics[width=0.5\linewidth]{fig/fig5.jpg}
% 	\caption{Схема экспериментальной установки}
% 	\label{fig:5}
% \end{figure}


\subsection{ВАХ диода при комнатной температуре}

% \subsection{Установка}
% Экспериментальная установка по наблюдению сигнала поглощения ЭПР собрана по схеме СВЧ-спектроскопии. Блок-схема радиоспектрометра ЭПР, используемого в данной работе, представленна на рис.1:
% \begin{figure}[h!]
% 	\centering
% 	\includegraphics[width=\linewidth]{fig/Sh.jpg}
% 	\caption{Блок-схема ЭПР-спектрометра}
% 	\label{fig:1}
% \end{figure}


% \begin{center}
% \begin{minipage}{0.45\linewidth}
%         \centering
%         \includegraphics[width=\linewidth]{fig/chi_p-2.jpg}  
%         \vspace{-20pt}
%         \label{fig:4}
%         \captionof{figure}{Кривая поглощения} 
%     \end{minipage} 
% \hfill     
%     \begin{minipage}{0.45\linewidth}
%         \includegraphics[width=\linewidth]{fig/chi_d-2.jpg} 
%         \vspace{-20pt}
%         \label{fig:3}
%         \captionof{figure}{Дисперсионная кривая} 
%       \end{minipage}
% \end{center} 

% Теоретический вид этих же кривых:
% \begin{figure}[H]
% 	\centering
% 	\includegraphics[width=\linewidth]{fig/norm.jpg}
% 	\caption{Нормированная кривая поглощения (а) и кривая дисперсии (б)}
% 	\label{fig:5}
% \end{figure}

% Здесь частота генератора $\nu=8.99$ ГГц. Сигнал появляется при значении тока $I=159 mA$, что соответствует полю в 3550 Гс и исчезает при $I=169 mA$, что соответствует полю 3750 Гс. Пересчет производится по графику зависимости напряженности магнитного поля $H_0$ от тока, протекающего через катушки электромагнита:
% \begin{figure}[H]
% 	\centering
% 	\includegraphics[width=0.9\linewidth]{fig/grad-3.jpg}
% 	\caption{Градуировочный график}
% 	\label{fig:6}
% \end{figure}

% Таким образом, для наблюдаемого сигнала вместо резонансного значения магнитного поля наблюдается целая полоса значений, соответствующая резонансу. 

% Далее измерение ширины линии поглощения сигнала ЭПР в единицах поля. 
% \begin{gather*}
%  	\Delta \omega=\frac{|e|\Delta H_0}{m_e c},
% \end{gather*}
% где $|e|=4.8 \cdot 10^{-13}$ед, $m=9 \cdot 10^{-31}$ кг, $c=3 \cdot 10^{10}$ см. 
% \begin{gather*}
%  	\Delta L=3 \text{кл}, \\
%  	\Delta I= |160-164| mA, \\
%  	\Delta H= 3625-3537.5=87.5 \text{Гс}, \\
%  	87.5 / 3 = 29.1, \\
%  	\delta H = 0.4 \cdot 29.1 = 11.7 \text{Гс}.
% \end{gather*}

% Ширина линии поглощения в единицах частоты
% \begin{gather*}
%  	\Delta \omega_{\text{прак}}=2.08 \cdot 10^{8} c^{-1}, \\
%  	\Delta T_{\text{2прак}} = 0.96 \cdot 10^{-8} c, \\
%  	\delta H_{\text{теор}}= 2.7 \text{Гс}, \\
%  	\Delta \omega_{\text{теор}}= 0.5 \cdot 10^{8} c^{-1}, \\
%  	\Delta T_{\text{2теор}} = 4 \cdot 10^{-8} c.
% \end{gather*}

% Определим число парамагнитных частиц. Для калибровки тракта усиления используется режим модуляции СВЧ-генератора меандром. Такой режим модуляции позволяет получить в схеме возбуждение резонатора переменным СВЧ-сигналом, амплитудный размах которого будет совпадать с реализованным ранее случаем непрерывного возбуждения. Таким образом, регистрация СВЧ-меандра как искусственно вводимого в систему эталонного переменного сигнала позволяет по существу прокалибровать вертикальную ось шкалы осциллографа и позволяет провести сравнение эталонного размаха меандра $L_M$ и наблюдаемого сигнала поглощения $L_C$. У нас: $L_C/L_M=2.3/84=0.027$. По формуле
% \begin{gather*}
%  	N_0=\frac{3kT}{8\pi Q \mu^2\omega_0 T_2^*}\frac{L_c}{L_M}\frac{P_M}{P_{\text{п}}}\frac{V_\text{рез}}{V_\text{обр}},
% \end{gather*}
% где $Q=5000, \frac{V_\text{рез}}{V_\text{обр}}=200, \frac{P_M}{P_{\text{п}}}=1, T=300 K, \omega_0=5.64 \cdot 10^{10} c^{-1}, \mu \sim 10^{-6}, T_2^* = 0.96 \cdot 10^{-8} c, k=1.38 \cdot 10^{-16}$ Дж/К. Получаем количество парамагнитных частиц
% \begin{gather*}
%  	N_0=9.8 \cdot 10^{-9}.
% \end{gather*}

% \subsection{ЭПР в кристалле рубина}
% Измерения проводятся на рубине с активными ионами $Cr^{3+}$. Здесь частота генератора $\nu=8.965$ ГГц. Сигнал появляется в диапазоне значений тока $\Delta I_1=151 - 139 ~mA$, что соответствует полям в 3350 - 3087 Гс и  $\Delta I_2=41 - 34~ mA$, что соответствует полям 925 - 800 Гс. Пересчет производится по графику зависимости напряженности магнитного поля $H_0$ от тока, протекающего через катушки электромагнита. Диапазон $\Delta I_1$ соответствует переходу $-1/2 \rightarrow 1/2$, $\Delta I_2$ - переходу $3/2 \rightarrow 1/2$.

% Далее измерение ширины линии поглощения сигнала ЭПР в единицах поля для перехода $-1/2 \rightarrow 1/2$. 
% \begin{gather*}
%  	\Delta L=2 \text{кл}, \\
%  	\Delta I= |48-44| mA, \\
%  	\Delta H= |1100-1012.5|=87.5 \text{Гс}, \\
%  	75 / 2 = 43.8, \\
%  	\delta H = 1 \cdot 43.8 = 43.8 \text{Гс},
% \end{gather*}
% и время поперечной релаксации
% \begin{gather*} 	
% 	\Delta \omega_{\text{прак}}=7.8 \cdot 10^{8} c^{-1}, \\
%  	\Delta T_{\text{2прак}} = 0.26 \cdot 10^{-8} c.
% \end{gather*}

% Отношение интенсивностей сигналов, соответствующих переходам $-1/2 \rightarrow 1/2$ и $3/2 \rightarrow 1/2$: $1.2/2=0.6$.

% Определим число парамагнитных частиц. Здесь У нас: $L_C/L_M=0.003/0.11=0.027$. $Q=5000, \frac{V_\text{рез}}{V_\text{обр}}=200, \frac{P_M}{P_{\text{п}}}=1, T=300 K, \omega_0=5.63 \cdot 10^{10} c^{-1}, \mu \sim 5 \cdot 10^{-7}, T_2^* = 0.26 \cdot 10^{-8} c, k=1.38 \cdot 10^{-16}$ Дж/К. Получаем количество парамагнитных частиц
% \begin{gather*}
%  	N_0=9.1 \cdot 10^{-9}.
% \end{gather*}

% \section{Заключение}
% В качестве заключения приведем ответ на вопрос 9 - найти теоретическое значение для отношения интенсивностей переходов $-1/2 \rightarrow 1/2$ и $3/2 \rightarrow 1/2$.
% \begin{gather*} 	
% 	E_{3/2} \rightarrow E_{1/2} \\
% 	\widehat{S_-}\chi_{3/2,1/2} = \hbar \sqrt{\frac32(\frac32+1)-\frac32(\frac32-1)}\chi_{3/2,1/2} = \sqrt{3} \hbar \chi_{3/2,1/2}; \\
% 	E_{-1/2} \rightarrow E_{1/2} \\
% 	\widehat{S_+}\chi_{3/2,-1/2} = \hbar \sqrt{\frac32(\frac32+1)+\frac12(-\frac12+1)}\chi_{3/2,1/2} = 2 \hbar \chi_{3/2,1/2}; \\
% 	\frac{I_{3/2 \rightarrow 1/2}}{I_{-1/2 \rightarrow 1/2}} = \frac{\sqrt{3}}{2}.
% \end{gather*}
\end{document}
